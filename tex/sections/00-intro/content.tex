\sectioncentered*{Введение}
\addcontentsline{toc}{section}{Введение}

Целью прохождения практики является формирование и развитие профессионального мастерства на основе изучения опыта работы конкретных организаций, учреждений, предприятий, привитие навыков самостоятельной работы будущим специалистам в условиях реально функционирующего производства.
Задачи практики:

\begin{itemize}
\item закрепление теоретических знаний, полученных студентами в университете;
\item изучение студентом производственно-экономической деятельности и системы управления той организации, которая определена в качестве места прохождения практики;
\item ознакомление студента с опытом работы персонала организации, занятого выработкой и принятием управленческих решений, составлением и реализацией, производственных заданий и программ, планов, а также выполнением других функций управления;
\item проверка степени готовности будущего специалиста к самостоятельной работе в условиях реального производства, выявление у студентов индивидуальных склонностей к практической работе на конкретных должностях, освоение несложных функциональных обязанностей на закрепленном за студентом по месту прохождения практики рабочем месте;
\item выявление у студентов способностей к исследовательской деятельности при проведении ими в организации, определенной в качестве места прохождения практики, простейших прикладных исследований конкретных производственных ситуаций и управленческих решений;
\item приобретение практических навыков анализа и прогноза социально-экономических процессов, обоснование управленческих решений в конкретных ситуациях.
\end{itemize}

Основные результаты и фактические материалы, полученные в период прохождения практики должны быть использованы студентом при написании дипломного проекта.