\subsubsection{} Обзор архитектуры iOS приложений
\label{sec:analysis:research:mobArch}

Любое современное мобильное приложение отображает динамически изменяемое содержимое, часто изменение одного поля модели способно повлечь за собой массу изменений в пользовательском интерфейсе. Важной проблемой при проектировании архитектуры мобильного приложения является консистентность состояния, данных и пользовательского интерфейса. Последние несколько лет сфера пытается уйти от императивного стиля программирования в сторону декларативного, для этого было разработано множество подходов, архитектур и инструментов. Одним из подходов является реактивное программирование и его конкретная имплементация -- \gls{rx}. Ниже будут рассмотрены традиционные подходы и инструменты, используемые при проектировании и имплементации архитектуры мобильного iOS приложения.

\paragraph {} Тест
Тест