\subsubsection{}
\label{sec:development:arch:ios:ufeature}

В пункте \ref{sec:analysis:research:mobArch:ufeature} был рассмотрен подход SoundCloud к организации модульных приложений, разработчики советуют использовать один репозиторий для всех модулей приложения, такой подход называется монорепозиторий и имеет следующие плюсы:

\begin{itemize}
\item консистентность версий --- конкретный набор версий компонент фиксируется не на уровне менеджера пакетов, а в пределах одного коммита;
\item гарантии компилируемости кода --- так как весь проект компилируется за раз, невозможно создать ситуацию, когда модули успешно компилируются по отдельности, но не работают вместе;
\item атомарность разработки --- изменения, происходящие в рамках разработки одной возможности, попадают в кодовую базу всех модулей одним коммитом;
\item скорость разработки --- отсутствие необходимости проходить несколько стадий пулреквестов и проверок работоспособности приложения сильно ускоряет процесс разработки на ранних этапах.
\end{itemize}

Данный подход выглядит идеальным на ранних стадиях разработки мобильного проекта, однако в реальности появляются проблеммы, связыанные со скоростью индексации и компиляции, отсуствии строгой изолированности и отсуствия поддержки со стороны инструментов разработки, невозможностью запускать изолированный набор тестов, поэтому было принято решение разбить проект на несколько репозиториев:

\begin{itemize}
\item chat.foundation.core --- набор основных сервисов приложения, например веб-клиент, криптография, логгер;
\item chat.foundation.ui --- имплементация \gls{mvvm} и переиспользуемые объекты домена View;
\item chat.product.* --- репозитории для каждой конкертной uFeature типа Product;
\item chat.dependency --- публичные интерфейсы сервисов из uFeature типа Product и набор всех Realm моделей;
\item chat.app --- репозиторий готового приложения.
\end{itemize}

Зависимости в репозиториях организованным следующим образом: 

\begin{itemize}
\item все модули уровня chat.foundation.* не имеют зависимостей;
\item chat.dependency имеет зависимости только на foundation;
\item chat.product.* имеют зависимости на depdendency(следовательно на foundation);
\item chat.app имеет зависимости на все модули приложения.
\end{itemize}

Каждый chat.product.* представляет из себя полноценный сценарий использования приложения, предоставляя в виде интерфейса начальные UIViewController и сервисы. Задачей приложения явяется правильным образом создать все product модули приложения и внедрить зависимости. Product модули не имеют возможности общаться друг с другом напрямую, поэтому на приложение ложится задача предоставления механима общения, который так же внедряется в каждый модуль при помощи Dependency Injection.g