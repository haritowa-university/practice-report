\subsubsection{}
\label{sec:analysis:research:crypto:general}

\emph{Криптография} --- наука о методах обеспечения конфиденциальности(невозможности прочтения информации посторонним), целостности данных(невозможности незаметного изменения информации), аутентификации(проверки подлинности авторства или иных свойств объекта), а также невозможности отказа от авторства. \cite{wiki:crypto}

Стоит ввести основные термины криптографии для дальнейшего обзора этой науки:

\begin{itemize}
	\item \emph{исходный текст} --- данные, находящиеся в исходном, не зашифрованном виде;
	\item \emph{шифр} --- семейство обратимых преобразований исходного теста в зашифрованный;
	\item \emph{ключ} --- параметр шифра, определяющий выбор конкретного преобразования данного текста;
	\item \emph{шифрование} --- процесс нормального применения криптографического преобразования открытого текста на основе алгоритма и ключа, в результате которого возникает шифрованный текст;
	\item \emph{расшифровывание} --- процесс нормального применения криптографического преобразования шифрованного текста в открытый;
	\item \emph{открытый ключ} --- тот из двух ключей асимметричной системы, который свободно распространяется;
	\item \emph{криптографический протокол} --- абстрактный или конкретный протокол, включающий набор криптографических алгоритмов. В основе протокола лежит набор правил, регламентирующих использование криптографических преобразований и алгоритмов в информационных процессах.
\end{itemize}

Криптографические примитивы:

\begin{enumerate}
	\item \emph{Симметричное шифрование}. Заключается в том, что обе стороны-участники обмена данными имеют абсолютно одинаковые ключи для шифрования и расшифровки данных. Данный способ осуществляет преобразование, позволяющее предотвратить просмотр информации третьей стороной.
	\item \emph{Асимметричное шифрование}. Предполагает использовать в паре два разных ключа — открытый и секретный. В асимметричном шифровании ключи работают в паре — если данные шифруются открытым ключом, то расшифровать их можно только соответствующим секретным ключом и наоборот — если данные шифруются секретным ключом, то расшифровать их можно только соответствующим открытым ключом. Использовать открытый ключ из одной пары и секретный с другой — невозможно. Каждая пара асимметричных ключей связана математическими зависимостями. Данный способ также нацелен на преобразование информации от просмотра третьей стороной.
	\item \emph{Хеширование}. Преобразование входного массива данных произвольной длины в выходную битовую строку фиксированной длины. Такие преобразования также называются хеш-функциями или функциями свёртки, а их результаты называют хеш-кодом, контрольной суммой или дайджестом сообщения (англ. message digest). Результаты хэширования статистически уникальны. Последовательность, отличающаяся хотя бы одним байтом, не будет преобразована в то же самое значение.
\end{enumerate}

Распространёнными алгоритмами шифрования являются:

\begin{itemize}
	\item симметричные: DES, AES, RC4 и др;
	\item асимметричные: RSA и Elgamal;
	\item хэш-функции: MD4, MD5, MD6, SHA-1, SHA-2.
\end{itemize}

Далее рассмотрены популярные алгоритмы шифрования, которые будут использованны в криптографическом протоколе разрабатываемого приложения.