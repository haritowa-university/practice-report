\subsubsection{}
\label{sec:analysis:research:mobArch:dataflow}

\textbf{Data flow} --- подход к программированию, при котором программа моделируется в виде ориентированного графа потока данных между операциями. \cite{wiki:data-flow} 

Традиционно программа моделируется как последовательность операций, происходящих в определённом порядке. Программирование потоков данных подчёркивает перемещение данных и рассматривает приложение как последовательность переходов. Явно определённые входы и выходы соединяют операции. Программы данного типа часто представлены в виде набора текстовых инструкций, которые описывают систему, последовательно передающую данные между небольшими инструментами, которые получают данные, модифицируют их и возвращают результат. 

\subsubsection{}
\label{sec:analysis:research:mobArch:reactive}

Реактивное программирование представляет собой асинхронную парадигму программирования, связанную с программированием потоков данных и распространением изменений. Языки с поддержкой парадигмы реактивного программирования позволяют описать статические и динамические потоки данных, имеют механизмы для уведомления подписчиков об изменениях. Например, выполнение операции \(a = c + b\) запишет в переменную \(a\) значение суммы \(b + c\). Последующее изменение \(b\) или \(c\) никак не отразится на переменной \(a\). В реактивном же программировании \(a\) будет обновлено при изменении любой переменной, которая участвует при вычислении значения \(a\).
Удобно выделять три типа программ:
\begin{enumerate}
	\item \emph{Программы трансформации} вычисляют результат для полученного набора входящих данных, типичным примером является компилятор или приложения для расчётов;
	\item \emph{Интерактивные программы} взаимодействуют с собственной скоростью с пользователем или другой программой;
	\item \emph{Реактивные программы} также поддерживают непрерывное взаимодействие с окружением, однако со скоростью, диктуемой этим окружением.
\end{enumerate}
Интерактивные программы работают в собственном темпе, когда реактивные могут работать только в ответ на внешние требования.