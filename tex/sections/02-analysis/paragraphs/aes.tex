\subsubsection{}
\label{sec:analysis:research:crypto:aes}

\emph{Advanced Encryption Standard} --- симметричный алгоритм блочного шифрования(размер блока 128 бит, ключ 128/192/256 бит), принятый в качестве стандарта шифрования правительством США по результатам конкурса AES. Этот алгоритм хорошо проанализирован и сейчас широко используется, как это было с его предшественником DES. \cite{wiki:aes}

По состоянию на 2009 год AES является одним из самых распространённых алгоритмов симметричного шифрования. \cite{thg:aes} Аппаратная поддержка AES(и только его) введена фирмой Intel в семейство процессоров x86 начиная с Intel Core i7-980X Extreme Edition, а затем на процессорах Sandy Bridge.

AES является стандартом, основанным на алгоритме Rijndael. Для AES длина input(блока входных данных) и State(состояния) постоянна и равна 128 бит, а длина шифроключа K составляет 128, 192, или 256 бит. При этом, исходный алгоритм Rijndael допускает длину ключа и размер блока от 128 до 256 бит с шагом в 32 бита. Для обозначения выбранных длин input, State и Cipher Key в 32-битных словах используется нотация Nb = 4 для input и State, Nk = 4, 6, 8 для Cipher Key соответственно для разных длин ключей.

В начале зашифрования input копируется в массив State по правилу \(state[r,c]=input[r+4c]\), для \(0 \leq r < 4\) и \(0 \leq c < Nb\). После этого к State применяется процедура AddRoundKey() и затем State проходит через процедуру трансформации(раунд) 10, 12, или 14 раз (в зависимости от длины ключа), при этом надо учесть, что последний раунд несколько отличается от предыдущих. В итоге, после завершения последнего раунда трансформации, State копируется в output по правилу \(output[r+4c]=state[r,c]\), для \(0 \leq r < 4\) и \(0 \leq c < Nb\).