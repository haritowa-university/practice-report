\subsubsection {}
\label{sec:analysis:research:backArch:microservices}

Микросервисная архитектура предлагает рассматривать приложение как набор слабо связанных сервисов, взаимодействующих друг с другом. Каждый сервис реализует фокусированный набор связанных функций. Например, приложение может содержать сервисы для авторизации, отправки сообщений, пуш-уведомлений.

Сервисы могут общаться при помощи как синхронных, так и асинхронных протоколов, часто выбор ложится на HTTP/REST. Такой подход позволяет разрабатывать и разворачивать сервисы независимо друг от друга, каждый сервис имеет собственную базу данных, консистентность данных между сервисами поддерживается при помощи паттерна Saga. \cite{microservices:ms}

На рисунке \ref{sec:analysis:research:arch:back:micro} представлен пример организации микросервисного приложения.

\begin{figure}[h]
  \centering
    \includegraphics[width=1\textwidth]{inc/img/backend-micro.png}
  \caption{Пример организации монолитного серверного приложения}
  \label{sec:analysis:research:arch:back:micro}
\end{figure}

Очевидными плюсами микросервисного подхода являются:

\begin{enumerate}
	\item \emph{Тестируемость} --- небольшие сервисы проще тестировать в изолированной среде.
	\item \emph{Скорость развёртывания} --- небольшой сервис разворачивается намного быстрее полноценного приложения.
	\item \emph{Модульность} --- микросервисная архитектура значительно упрощает разделение сотрудников на команды.
	\item \emph{Меньший порог входа} --- разработчику нужно разбираться только с одним небольшим сервисом.
	\item \emph{Скорость разработки} --- отсутствие неявных зависимостей, скорость развёртывания и скорость работы \gls{ide} положительно сказываются на продуктивности разработчиков.
	\item \emph{Повышенная отказоустойчивость} --- выход из строя одного сервиса не выведет из строя всё приложение.
	\item \emph{Гибкость масштабирования} --- разработчики получают возможность масштабировать отдельно сервисы.
	\item \emph{Гибкость в выборе технологического стека} --- каждый сервис может быть разработан, используя собственный технологический стек, переход на новую технологию можно выполнять поэтапно.
\end{enumerate}

Однако, микросервисная архитектура значительно повышает общую сложность системы, вводя дополнительную работу для программистов:

\begin{enumerate}
	\item \emph{Отсутствие поддержки со стороны \gls{ide}} --- если перед разработчиком стоит задача, которая предполагает изменение нескольких сервисов --- \gls{ide} будет неспособна работать над несколькими проектами одновременно.
	\item Появляется необходимость разработки и поддержки протокола общения между сервисами.
	\item Перед разработчиками появляются задачи координации сервисов, аггрегации запросов.
	\item \emph{Усложнённый процесс запуска всего приложения} --- для запуска всего приложения требуется развернуть каждый сервис отдельно.
	\item \emph{Повышенное потребление ресурсов} --- каждый сервис является отдельным процессом, поэтому ресурсы, которые тратятся на поддержку работоспособности процесса увеличиваются на количество сервисов.
\end{enumerate}

% Так как каждый сервис имеет собственную базу данных, а часто бизнес-логика требует работы сразу нескольких сервисов, стоит рассмотреть механизмы для аггрегации запросов между сервисами и поддержки консистентности состояния.

% -- Api Composition
% -- Saga pattern